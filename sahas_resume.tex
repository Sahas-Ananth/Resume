%! TeX program = xelatex
\documentclass{resume} % Use the custom resume.cls style
\usepackage{hyperref}
\usepackage{amsmath}
\usepackage{fancyvrb}

\hypersetup{
    colorlinks=true,
    linkcolor=blue,
    filecolor=magenta,      
    urlcolor=blue,
}

\usepackage[left=0.4in,top=0.1in,right=0.4in,bottom=0.4in]{geometry} % Document margins
\newcommand{\tab}[1]{\hspace{.2667\textwidth}\rlap{#1}}
\newcommand{\itab}[1]{\hspace{0em}\rlap{#1}}

\begin{document}
\setmainfont{Arial}
\begin{center}
	\textbf{\LARGE Sahasrajit Anantharamakrishnan}\\
	\vspace{1ex}
	\begin{tabular}{c c c c}
		\href{mailto:anantharamakrishn.sa@northeastern.edu}{\small anantharamakrishn.sa@northeastern.edu} &
		% \href{mailto:sahas_ananth@outlook.com}{sahas\_ananth@outlook.com}              &
		\href{tel:18572062833}{+1 (857) 206-2833}                                                         &
		\href{https://www.linkedin.com/in/sahas-ananth/}{linkedin.com/in/sahas-ananth}                    &
		\href{https://sahas-ananth.github.io/}{sahas-ananth.github.io}
		% \href{https://github.com/Sahas-Ananth}{GitHub(Sahas-Ananth)} &
		% \href{https://goo.gl/maps/rhjbcRKJK2ePYaUHA}{Boston, MA}
	\end{tabular}
\end{center}

% \textit{\hspace{8mm}I am proactive about self-learning and gaining familiarity with new programs and skills. As such, I am able to independently work on projects and believe in ensuring a high standard of work which entails good coding practices and scalability in code. Furthermore, I am also a great team-player and appreciate the value of collaboration. In teams, I am able to assume leadership roles and maintain good teamwork environments by encouraging effective communication and motivating colleagues.}

% \textit{\hspace{8mm}I'm a proactive learner and highly adaptable, with a track record of success in tackling new programs and technologies. As a graduate student at Northeastern University studying Robotics Engineering, I bring a strong foundation in this field to every project. I'm also a team player, skilled at fostering a collaborative atmosphere and stepping up to lead when needed. Above all, I'm committed to delivering top-notch results that meet the highest standards of coding practices and scalability.}

% EDUCATION SECTION
\begin{rSection}{EDUCATION}
	\rEducation{Northeastern University}{Boston, MA}{May 2024}{Master of Science in Robotics Engineering}{3.939/4.000}{Legged Robotics, Graph Theory, Deep Learning, Autonomous Field Robotics, Mobile Robotics, Computer Vision, Reinforcement Learning \& Sequential Decision Making}

	\rEducation{Anna University}{Chennai, India}{May 2022}{Bachelor of Engineering in Electrical and Electronics Engineering}{8.66/10.00}{}%{Digital Signal Processing, Embedded Systems, Robotics \& Machine Vision System, Control Systems, and Data Structures}\\
\end{rSection}
\vspace{-5mm}

% WORK EXPERIENCE SECTION
\begin{rSection}{WORK EXPERIENCE}
	\begin{rProjExpDetails}{Robotics and Intelligent Vehicles Research Laboratory (RIVeR)}{Boston, MA}{May 2024 - Present}{Robotics Research Assistant, \textbf{Project:} Stochastic Model Predictive Control for bipedal loco-manipulation}{}{}
		\item Introduced probabilistic models into traditional control algorithm, MPC to create Stochastic MPC (SMPC), to improve adaptability and robustness against uneven terrain and unexpected loads.
		\item Spearheaded the adaptation of the SMPC framework from quadrupedal to bipedal robots to demonstrate its generalizability, using simulation platforms such as PyBullet and Gazebo
		\item Refined the dynamics model and cost function to satisfy the constraints of the bipedal robot to guarantee stability % and efficiency
	\end{rProjExpDetails}
	\begin{rProjExpDetails}{Northeastern Autonomy and Intelligence Laboratory (NAIL)}{Boston, MA}{January 2023 - May 2024}{Robotics Research Assistant, \textbf{Project:} High-Speed Off-Road Autonomy Robot}{https://neu-autonomy.github.io/lab_website/team/}{Lab Link}
		\item Developed an innovative 2.5D terrain mapping model accommodating uncertainties in both the shape and properties of challenging off-road environment
		\item Created and Optimized a custom MPPI algorithm using JAX python, slashing average run time from \hl{1000 ms} to \hl{1 ms}
		\item Crafted a custom cost function for MPPI controls, prioritizing speed in unstructured environments while considering the robot's kino-dynamics, terrain traversability, and safety constraints
		\item Fine-tuned STEGO perception algorithm, a self-supervised semantic segmentation head for DINOv1 vision transformer, on RUGD, RELLIS, and a custom dataset to achieve clear class clusters for RGB image semantic segmentation
		% \vspace{-2mm}
		% \item[-] Increased robustness and adaptability to diverse environments through self-supervised learning
		% \item[-] Planned future incorporation of online learning in STEGO to increase robustness to unknown objects.
		% \end{itemize}
		% \item Implemented STEGO, a self-supervised semantic segmentation AI model, utilizing DINOv1 Vision Transformer as its backbone. Addressed the curse of dimensions problem, resulting in clear semantic class clusters for RGB image segmentation.
		% \item Facilitated deployment in diverse environments by utilizing self-supervised learning of STEGO thereby reducing reliance on human annotation, and enhancing the adaptability and robustness of the algorithm.
		% \item Designed and fabricated a custom high-speed, off-road mobile robot using Fusion 360, to enable seamless customization
		% \item Implemented sensor fusion in ROS between 3D-LiDAR and semantic image, obtaining semantic point cloud for perception
		\item Employed sensor fusion techniques to combine 3D-LiDAR data with semantically segmented RGB images, resulting in a Semantic Point Cloud, essential for downstream perception, control, and motion planning tasks
		% \item Improved the Direct LiDAR-Inertial Odometry (DLIO) SLAM algorithm to accept semantic point cloud as additional input, enabling the integration of semantic information derived from the vision transformer model
		% \item Integrated SLAM (Simultaneous Localization and Mapping) and autonomous navigation algorithms, in both ROS1 and ROS2, enabling high-speed off-road autonomy capability in mobile robots
		\item Utilized Fusion 360 to engineer and assemble a customized compute and sensor suite payload, designed to meet the distinct needs of AgileX's scout and Clearpath's Warthog robotic platforms, to enable high-speed offroad autonomy capability
		% \item Developed a novel Vision Transformer-based Deep Neural Network for Semantic Segmentation of RGB/RGB-D images.
		% \item Incorporated Unsupervised and online -learning techniques to enhance the network's robustness.
		% \item Designed and Fabricated a custom Mobile Robot with High-Speed, Off-Road Autonomy capabilities.
		% \item Implemented sensor fusion between 3D-LiDAR and camera to obtain precise RGB-D images.
	\end{rProjExpDetails}

	\begin{rProjExpDetails}{Rigbetal Labs LLP}{Pune, India}{August 2021 - November 2021}{Robotics Engineer Intern}{}{}
		\item Formulated a novel algorithm, Road Anomaly Detection System (RADS), in C$++$ to detect road anomalies (Potholes, Speed Bumps, etc.) using normal estimation
		\item Reduced cost by 90\%, by generating a 3D Point cloud from a series of moving 2D Laser scans
		\item Simulated a multi-agent (robot) mapping environment in Gazebo ROS to create a cohesive 2D map
		\begin{itemize}
			\item Deployed the same in a cloud environment using AWS Robomaker to enable remote multi-user control of agents
		\end{itemize}
	\end{rProjExpDetails}

	\begin{rProjExpDetails}{Capgemini Technologies Services}{Bangalore, India}{July 2020 - December 2020}{Robotics (Medical Devices) Intern}{https://github.com/Sahas-Ananth/ROS-ASV}{}
		\item Fabricated a ROS-based autonomous ground vehicle in Fusion 360 to sterilize and sanitize offices from SARS-COV2 virus with Ultraviolet (UV-C) irradiation
		\item Led communication and task allocation for cross-functional teams and clients, boosting team efficiency and client relations
	\end{rProjExpDetails}
\end{rSection}
% \vspace{-3mm}
\begin{rSection}{SKILLS}
	\begin{tabular}{ @ {} >{\bfseries}l @{\hspace{3ex}} l }
		Languages / Libraries & Python, PyTorch, JAX, C$++$, CUDA, C, MATLAB, OpenCV, Tensorflow, PCL                     \\
		Software and Tools    & ROS, Ubuntu Linux, Git, CMake, Docker, Gazebo, Nvidia Issac Sim, PyBullet, MQTT, Simulink \\
		                      & Fusion 360, Blender, LaTeX                                                                \\
	\end{tabular}
\end{rSection}
\vspace{-3.0mm}
\begin{rSection}{PROJECTS}
	% TODO: Add SLAM project.
	% \begin{rProjExpDetails}{Stochastic Model Predictive Control for quadrupedal and bipedal loco-manipulation}{}{April 2024 - Present}{Stochastic MPC (SMPC) introduces probabilistic models to MPC, improving robustness against uncertainties like variable terrain and unexpected loads in robot dynamics}{}{}
	% 	% \item Stochastic MPC (SMPC) introduces probabilistic models to MPC, improving robustness against uncertainties like variable terrain and unexpected loads
	% 	\item Spearheaded the adaptation of the SMPC framework to bipedal robots in simulation such as PyBullet and Gazebo
	% 	\item Refined mathematical models and cost functions to optimize bipedal robot stability and efficiency %, preparing foundational research for publication.
	% \end{rProjExpDetails}

	\begin{rProjExpDetails}{Implementing Batch Informed Trees (BIT*) Motion planning Algorithm}{}{March 2023 - April 2023}{Paper: \href{https://journals.sagepub.com/doi/pdf/10.1177/0278364919890396}{\textit{Batch Informed Trees (BIT*): Informed asymptotically optimal anytime search}}}{https://sahas-ananth.github.io/projects/BIT_star.html}{}
		\item Reduced the run-time of the algorithm in python using hash-maps, parallelization, and caching
		\item Engineered intuitive visualization techniques to better analyze the BIT* algorithm
		\item Tested and validated the algorithm against baselines results such as RRT, RRT*, FMT*, and RRT Connect
	\end{rProjExpDetails}
	%
	% \begin{rProjExpDetails}{Learning Inverse Kinematics using Reinforcement Learning}{}{October 2022 - December 2022}{A 7 DoF robot arm which will reach a goal location trained with Reinforcement Learning}{https://github.com/Sahas-Ananth/RL-FinalProject}{}
	% 	\item Implemented and evaluated Deep Deterministic Policy Gradients (DDPG), Twin Delayed Deep Deterministic Policy Gradients (TD3), and Soft Actor-Critic (SAC) algorithms, with TD3 demonstrating the best performance.
	% \end{rProjExpDetails}
	%
	% \begin{rProjExpDetails}{Comparative Analysis of Cartographer and ORB SLAM Algorithms}{}{November 2022 - December 2022}{Comparing two different SLAM algorithms, Cartographer (3D) and ORB SLAM 3 on the NUance data set.}{}{}
	% 	\item Developed and Analysed ORB-SLAM3 on the NUance data set.
	% \end{rProjExpDetails}
	%
	% \begin{rProjExpDetails}{Intelligent Quads, \texttt{iq\_gnc} (an Open source Project)}{}{June 2021}{An Ardupilot and ROS based quadrotor project with a Guidance and Navigation System}{https://github.com/Intelligent-Quads/iq_gnc}{}
	% 	\item Converted project from C$++$ to Python to make it more beginner-friendly
	% 	\item Setup C.I. Pipeline in GitHub actions
	% \end{rProjExpDetails}
	%
	% \begin{rProjExpDetails}{Vargi Bots - e-Yantra Robotics Competition}{}{November 2020 - March 2021}{Simulated two robot arms to sort coloured boxes simultaneously according to their priority from a conveyor}{https://github.com/Sahas-Ananth/Vargi_Bot_Eyantra}{}
	% 	\item Utilized MQTT (IoT) for receiving orders and for status updates.
	% 	\item OpenCV was used to recognize the color of the boxes and reading the QR codes.
	% 	\item ROS was used to control the two robot arms.
	% \end{rProjExpDetails}
	%
	% \begin{rProjExpDetails}{Autonomous In-campus Drone Delivery}{}{June 2020}{A fully autonomous delivery drone that delivers essential and nonessential products across campus.}{}{}
	% 	\item Designed the drone in CAD software
	% 	\item Chief Architecture Officer of the software subsystem.
	% \end{rProjExpDetails}

\end{rSection}

% \begin{rSection}{ACHIEVEMENTS}
% \begin{rSubsectiond}{Internship Winner}{\textit{July 2021}\\Indian Robotics Community Knowledge Transfer (IRCKT) Competition with a series of tasks based on ROS.}
% \end{rSubsectiond}
%
% \begin{rSubsectiond}{\textbf{Best paper award - International Conference Presentation EDGE 2020.}}{\textbf{March 2020}}
% \item Refilling a liquid-based optical lens system using a microfluidic pump and channel system in MEMS
% \end{rSubsectiond}
% \begin{rSubsectiond}{\textbf{Best paper award - International Conference Presentation EDGE 2020.}}{\textbf{March 2020}}
% \item Diagnosis Of Hypertrophic Cardiomyopathy Using ORB Image Matching In Python
% \end{rSubsectiond}
% \end{rSection}
%
% \begin{rSection}{RESPONSIBILITIES}
% \begin{rSubsectiond}{Robotics, Automation \& AI Project Head}{\textit{November 2020 - November 2021}\\Designers Consortium Club, Rajalakshmi Engineering College}
% \end{rSubsectiond}
% \end{rSection}
\end{document}
