%! TeX program = xelatex
\documentclass{resume} % Use the custom resume.cls style
\usepackage{hyperref}
\usepackage{amsmath}
\usepackage{fancyvrb}

\hypersetup{
    colorlinks=true,
    linkcolor=blue,
    filecolor=magenta,      
    urlcolor=blue,
}

\usepackage[left=0.4in,top=0.4in,right=0.4in,bottom=0.4in]{geometry} % Document margins
\newcommand{\tab}[1]{\hspace{.2667\textwidth}\rlap{#1}}
\newcommand{\itab}[1]{\hspace{0em}\rlap{#1}}

\begin{document}
\setmainfont{Arial}
\vspace{-0.4in}
\begin{center}
	\textbf{\LARGE Sahasrajit Anantharamakrishnan}\\
	\vspace{1ex}
	\begin{tabular}{c c c c}
		\href{mailto:sahas_ananth@outlook.com}{sahas\_ananth@outlook.com}              &
		\href{tel:18572062833}{+1 (857) 206-2833}                                      &
		\href{https://www.linkedin.com/in/sahas-ananth/}{linkedin.com/in/sahas-ananth} &
		\href{https://sahas-ananth.github.io/}{sahas-ananth.github.io}
	\end{tabular}
\end{center}

\textit{Full-Stack Robotics Engineer with expertise in hardware design, perception, planning, and control. Specializing in motion planning, classical, and learning-based control of mobile robots, manipulators, and legged systems. Skilled in C++, Python, CUDA, and ROS, with experience in GPU acceleration, real-time perception, and system integration.}

% EDUCATION SECTION
\begin{rSection}{EDUCATION}
	\rEducation{Northeastern University}{Boston, MA}{May 2024}{Master of Science in Robotics}{3.939/4.000}{}%{Legged Robotics, Graph Theory, Deep Learning, Autonomous Field Robotics, Mobile Robotics, Computer Vision, Reinforcement Learning \& Sequential Decision Making}

	\vspace{-2mm}

	\rEducation{Anna University}{Chennai, India}{May 2022}{Bachelor of Engineering in Electronics and Communication Engineering}{8.66/10.00}{}%{Digital Signal Processing, Embedded Systems, Robotics \& Machine Vision System, Control Systems, and Data Structures}\\
\end{rSection}

\vspace{-2mm}

\begin{rSection}{SKILLS}
	\begin{tabular}{ @ {} >{\bfseries}l @{\hspace{3ex}}p{0.76\textwidth}}
		Languages / Libraries & Python, C++, Drake, IPOPT, SNOPT, Gurobi, C, CUDA, PyTorch, JAX, Tensorflow, MATLAB \& Simulink, OMPL                                                                                                     \\
		Software and Tools    & ROS, Ubuntu Linux, Git, CMake, Docker, Gazebo, Nvidia Issac Sim, PyBullet, MQTT,
		Fusion 360, Blender, CI/CD, Automated Testing, gRPC, Protobuf, MuJoCo                                                                                                                                                             \\
		Proficient Concepts   & Optimization, Motion planning, Control systems, Trajectory Optimization, Kinematics \& Dynamics Modeling, Deep Learning, Reinforcement Learning, Machine Learning, Computer Vision, State estimation/SLAM \\
	\end{tabular}
\end{rSection}

\vspace{-3mm}

% WORK EXPERIENCE SECTION
\begin{rSection}{WORK EXPERIENCE}
	\begin{rProjExpDetails}{Robotics and Intelligent Vehicles Research Laboratory (RIVeR)}{Boston, MA}{May 2024 - Present}{Robotics Research Assistant, \textbf{Project:} Stochastic Model Predictive Control for bipedal loco-manipulation}{}{}
		\item Improved robustness against uncertainties in terrain and payload of Bipedal robot by adding soft-constraints to Model Predictive Control (MPC)
		\item Created novel simulation environments in NVIDIA Issac Sim and PyBullet, for simulating, and testing the humanoid robot
		\item Improved the constraints and the dynamics model to guarantee stability
	\end{rProjExpDetails}
	\begin{rProjExpDetails}{Northeastern Autonomy and Intelligence Laboratory (NAIL)}{Boston, MA}{January 2023 - May 2024}{Robotics Research Assistant, \textbf{Project:} High-Speed Off-Road Autonomy Robot}{https://neu-autonomy.github.io/lab_website/team/}{Lab Link}
		\item Set up the lab from scratch, showcasing full-stack expertise in hardware, perception, planning, and control
		\item[] \underline{\textbf{Motion Planning and Control:}}
		\item Optimized the trajectory of AVs using a custom Model Predictive Path Integral (MPPI) algorithm, a costmap-based planner
		\item Formulated a custom cost function for the robot to account for kinematics \& dynamic constraints, and terrain traversability
		\item Optimized algorithm runtime using GPU programming with CUDA C++ and JAX by 1000x
		\item[] \underline{\textbf{Perception:}}
		\item Fine-tuned, using PyTorch, a Vision Transformer AI model on a custom dataset to semantically segment rough terrain
		\item Increased training ease by 38\% using SLURM and Docker to train, and run inference on GPU server cluster
		\item Created a custom data pipeline for 50,000 RGB images that included data collection, logging, storage and training
		\item Monitored model training progress and performance using TensorBoard and Weights and Biases (WandB)
		% \item Sensor fused a 3D-LiDAR and semantically segmented RGB images to create a 2.5D map used for navigation tasks
		\item[] \underline{\textbf{Hardware:}}
		\item Fabricated a custom mobile robot using Fusion 360 to be used in high-speed off-road environments
		\item Engineered a custom electrical and network system of the robot to ensure safety and reliability
	\end{rProjExpDetails}

	\begin{rProjExpDetails}{Rigbetal Labs LLP}{Pune, India}{August 2021 - November 2021}{Robotics Engineer Intern}{}{}
		\item Formulated a novel algorithm, Road Anomaly Detection System (RADS), in C++ to detect road anomalies
		\item Reduced cost by 90\%, by generating a 3D Point cloud from a series of moving 2D Laser scans
		\item Developed a custom multi-agent path planning and mapping framework in AWS Robomaker and Gazebo
		% \item Tested and Deployed the code using a custom CI/CD pipeline to ensure safe and reliable code
	\end{rProjExpDetails}

	% \begin{rProjExpDetails}{Capgemini Technologies Services}{Bangalore, India}{July 2020 - December 2020}{Robotics (Medical Devices) Intern}{https://github.com/Sahas-Ananth/ROS-ASV}{}
	% 	\item Fabricated a ROS-based autonomous ground vehicle in Fusion 360 to sterilize and sanitize offices from SARS-COV2 virus with Ultraviolet (UV-C) irradiation
	% 	\item Led communication and task allocation for cross-functional teams and clients, boosting team efficiency and client relations
	% \end{rProjExpDetails}
\end{rSection}
\begin{rSection}{PUBLICATIONS}
	\nocite{*}
	\printbibliography[heading=none]
\end{rSection}
\begin{rSection}{PROJECTS}
	\begin{rProjExpDetails}{Implementing Batch Informed Trees (BIT*) Motion planning Algorithm for Robot Arms}{}{March 2023 - April 2023}{Paper: \href{https://journals.sagepub.com/doi/pdf/10.1177/0278364919890396}{\textit{Batch Informed Trees (BIT*): Informed asymptotically optimal anytime search}}}{https://sahas-ananth.github.io/projects/BIT_star.html}{}
		\item Increased run-time efficiency of the algorithm by using hash-maps, parallelization and caching
		\item Engineered intuitive visualization and analysis tools to validate the algorithm
		\item Tested and validated the algorithm against baselines algorithms such as RRT, RRT*, FMT*, and RRT Connect
	\end{rProjExpDetails}

	% \begin{rProjExpDetails}{Learning Inverse Kinematics using Reinforcement Learning}{}{October 2022 - December 2022}{A 7 DoF robot arm which will reach a goal location trained with Reinforcement Learning}{https://github.com/Sahas-Ananth/RL-FinalProject}{}
	% 	\item Implemented AI-driven planning algorithms: DDPG, TD3, PPO, and SAC, with TD3 achieving the best performance
	% 	% \item Implemented and evaluated Deep Deterministic Policy Gradients (DDPG), Twin Delayed Deep Deterministic Policy Gradients (TD3), Proximal Policy Optimization (PPO) and Soft Actor-Critic (SAC) algorithms , with TD3 demonstrating the best performance.
	% \end{rProjExpDetails}
	%
	% \begin{rProjExpDetails}{Comparative Analysis of Cartographer and ORB SLAM Algorithms}{}{November 2022 - December 2022}{Comparing two different SLAM algorithms, Cartographer (3D) and ORB SLAM 3 on the NUance data set.}{}{}
	% 	\item Developed and Analysed ORB-SLAM3 on the NUance data set.
	% \end{rProjExpDetails}
	%
	% \begin{rProjExpDetails}{Intelligent Quads, \texttt{iq\_gnc} (an Open source Project)}{}{June 2021}{An Ardupilot and ROS based quadrotor project with a Guidance, Navigation, and Control (GNC) System}{https://github.com/Intelligent-Quads/iq_gnc}{}
	% 	\item Converted project from C$++$ to Python to make it more beginner-friendly
	% 	\item Setup C.I. Pipeline in GitHub actions
	% \end{rProjExpDetails}
	%
	% \begin{rProjExpDetails}{Vargi Bots - e-Yantra Robotics Competition}{}{November 2020 - March 2021}{Controlled two robot arms to sort coloured boxes simultaneously according to their priority from a conveyor}{https://github.com/Sahas-Ananth/Vargi_Bot_Eyantra}{}
	% 	\item Motion planned and Controlled robot dual-arms using ROS
	% 	\item Utilized Computer Vision algorithms with OpenCV to recognize the color of the boxes and reading the QR codes.
	% 	% \item Utilized MQTT (IoT) for receiving orders and for status updates.
	% \end{rProjExpDetails}
	%
	% \begin{rProjExpDetails}{Autonomous In-campus Drone Delivery}{}{June 2020}{A fully autonomous delivery drone that delivers essential and nonessential products across campus.}{}{}
	% 	\item Designed the drone in CAD software
	% 	\item Chief Architecture Officer of the software subsystem.
	% \end{rProjExpDetails}

\end{rSection}
\end{document}
